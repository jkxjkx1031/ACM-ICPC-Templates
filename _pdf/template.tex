\documentclass[a4paper,12pt]{article}
\usepackage{listings,color,geometry}
\usepackage[colorlinks,linkcolor=black,citecolor=black,urlcolor=blue]{hyperref}


\geometry{left=3cm,right=3cm,top=3cm,bottom=3cm}


\definecolor{grey}{rgb}{0.8,0.8,0.8}
\definecolor{blue}{rgb}{0,0,1}
\definecolor{green}{rgb}{0,0.6,0}
\lstset{
language=C++,
frame=shadowbox,
basicstyle=\ttfamily\footnotesize,
keywordstyle=\bfseries\color{blue},
commentstyle=\slshape\color{green},
breaklines=true,
framexleftmargin=3pt,framexrightmargin=3pt,framextopmargin=3pt,framexbottommargin=3pt,
xleftmargin=2em
}


\begin{document}
\tableofcontents
\newpage


\section{Math}
\subsection{Quick power}
\begin{lstlisting}
int power(int x,int e)
{
    if(!e) return 1;
    LL res=power(x,e/2);
    return (e%2 ? res*res%M*x%M : res*res%M);
}
\end{lstlisting}

\subsection{Matrix multiply}
\begin{lstlisting}
struct MAT
{
    int v[SZ][SZ];
    int sz;
    MAT()
    {
        memset(v,0,sizeof(v));
    }
    MAT(int s,int k)
    {
        sz=s;
        int i;
        if(k==0) memset(v,0,sizeof(v));
        else if(k==1)
        {
            memset(v,0,sizeof(v));
            for(i=0;i<sz;i++) v[i][i]=1;
        }
    }
    MAT operator*(const MAT &rhs) const
    {
        int i,j,k;
        MAT tmp(sz,0);
        for(i=0;i<sz;i++)
            for(j=0;j<sz;j++)
                for(k=0;k<sz;k++)
                    tmp.v[i][j]+=v[i][k]*rhs.v[k][j];
        return tmp;
    }
    MAT operator^(int e) const
    {
        int i;
        MAT tmp(sz,1);
        for(i=1;i<=e;i++)
            tmp=tmp*(*this);
        return tmp;
    }
};

/*To multiply a matrix by a vector, first make the vector into a square matrix!*/
\end{lstlisting}

\subsection{Matrix quick power}
\begin{lstlisting}
MAT mpw(MAT m,int e,int sz)
{
    if(!e) return MAT(sz,1);
    MAT t=mpw(m,e/2,sz);
    if(e%2) return t*t*m;
    else return t*t;
}
\end{lstlisting}

\subsection{Primes}
\begin{lstlisting}
#include <stdio.h>
#include <string.h>
#define N 50000000

typedef long long LL;
LL f[N+10];

int main(void)
{
    freopen("prime.txt","w",stdout);
    memset(f,0,sizeof(f));
    LL i,j,prod=1;
    for(i=2;i<=N;i++)
        if(!f[i])
        {
            if(i<=60)
            {
                prod*=i;
                printf("%lld    prod: %lld\n",i,prod);
            }
            else printf("%lld\n",i);
            for(j=i;i*j<=N;j++) f[i*j]=1;
        }
    return 0;
}
\end{lstlisting}

\subsection{Moebius function}
\begin{lstlisting}
void moebius(int n)
{
    vector<int> p;
    int i,j,t;
    for(t=n,i=2;i*i<=n;i++)
    {
        if(t%i==0) p.push_back(i);
        while(t%i==0) t/=i;
    }
    if(t>1) p.push_back(t);
    int m=p.size(),x,y;
    for(mu.clear(),i=0;i<1<<m;i++)
    {
        for(x=1,y=1,j=0;j<m;j++)
            if(i&(1<<j))
            {
                x*=p[j];
                y*=-1;
            }
        mu[x]=y;
    }
}

/*  count of non-periodic strings:

#include <cstdio>
#include <cstring>
#include <algorithm>
#include <map>
#include <vector>
using namespace std;
typedef long long LL;
const int M=1000000007;
map<int,int> mu;
int T,n;
void moebius(int n);
int power(int x,int e);

int main()
{
    scanf("%d",&T);
    while(T--)
    {
        scanf("%d",&n);
        moebius(n);
        int ans=0;
        map<int,int>::iterator iter;
        for(iter=mu.begin();iter!=mu.end();++iter)
            ans=((ans+power(10,n/iter->first)*iter->second)%M+M)%M;
        printf("%d\n",ans);
    }
    return 0;
}

int power(int x,int e)
{
    if(!e) return 1;
    LL res=power(x,e/2);
    return (e%2 ? res*res%M*x%M : res*res%M);
}

void moebius(int n)
{
    vector<int> p;
    int i,j,t;
    for(t=n,i=2;i*i<=n;i++)
    {
        if(t%i==0) p.push_back(i);
        while(t%i==0) t/=i;
    }
    if(t>1) p.push_back(t);
    int m=p.size(),x,y;
    for(mu.clear(),i=0;i<1<<m;i++)
    {
        for(x=1,y=1,j=0;j<m;j++)
            if(i&(1<<j))
            {
                x*=p[j];
                y*=-1;
            }
        mu[x]=y;
    }
}
*/
\end{lstlisting}

\subsection{Linear basis}
\begin{lstlisting}
int l,b[BIT_LENGTH];
// l (maximal bit length) needs to be initialized; b[] should be set to 0 initially

void insert(int x)
{
    int i,j;
    for(i=l-1;i>=0;i--)
    {
        if(!(x&(1<<i))) continue;
        if(b[i]) x^=b[i];
        else
        {
            for(j=0;j<i;j++) x^=b[j];
            for(j=i+1;j<l;j++)
                if(b[j]&(1<<i)) b[j]^=x;
            b[i]=x;
            break;
        }
    }
}

int find(int x)
{
    int i;
    for(i=l-1;i>=0;i--)
        if(x&(1<<i)) x^=b[i];
    return x ? 0 : 1;
}
\end{lstlisting}

\subsection{Fast Fourier Transform}
\begin{lstlisting}
void build(const CPLX x[],int &i,int s,int d,int l,CPLX x2[])
{
    if(l==1)
    {
        x2[s]=x[i++];
        return;
    }
    build(x,i,s,d*2,l/2,x2);  build(x,i,s+d,d*2,l/2,x2);
}

void fft(const CPLX x[],int n,CPLX y[],int o)  // o=1: FFT;  o=-1: Inverse FFT 
{
    int t=0;
    build(x,t,0,1,n,y);
    int i,j,k;
    CPLX t1,t2,w,w0;
    for(i=1;i<n;i*=2)
        for(j=0;j<n;j+=i*2)
            for(k=0,w=1,w0=polar(1.0,PI/i*o);k<i;k++,w=w*w0)
            {
                t1=y[j+k];  t2=w*y[j+k+i];
                y[j+k]=t1+t2;  y[j+k+i]=t1-t2;
            }
    if(o>0) return;
    for(i=0;i<n;i++) y[i]/=n;
}
\end{lstlisting}

\subsection{Simplex}
\begin{lstlisting}
int id[N_ROW+N_COL];

void pivot(double a[][N_COL],int x,int y,int r,int c)
{
    int i,j;
    double t;
    std::swap(id[x+c],id[y]);
    t=-a[x][y];  a[x][y]=-1;
    for(i=0;i<=c;i++) a[x][i]/=t;
    for(i=0;i<=r;i++)
    {
        if(i==x) continue;
        t=a[i][y];  a[i][y]=0;
        for(j=0;j<=c;j++)
            a[i][j]+=t*a[x][j];
    }
}

double simplex(double a[][N_COL],double v[],int r,int c)
{
    int i,x,y;
    double t;
    for(i=1;i<=r+c;i++) id[i]=i;
    for(;;)
    {
        for(x=0,i=1;i<=r;i++)
            if(a[i][0]<-EPS)
            {
                x=i;
                break;
            }
        if(!x) break;
        for(y=0,i=1;i<=c;i++)
            if(a[x][i]>EPS)
            {
                y=i;
                break;
            }
        if(!y) return NAN;  // Unfeasible
        pivot(a,x,y,r,c);
    }
    for(;;)
    {
        for(y=0,i=1;i<=c;i++)
            if(a[0][i]>EPS)
            {
                y=i;
                break;
            }
        if(!y) break;
        for(x=0,t=INFINITY,i=1;i<=r;i++)
            if(a[i][y]<-EPS && -a[i][0]/a[i][y]<t)
                x=i,t=-a[i][0]/a[i][y];
        if(!x) return NAN;  // Unbounded
        pivot(a,x,y,r,c);
    }
    for(i=1;i<=c;i++)
        if(id[i]<=c) v[id[i]]=0;
    for(i=1;i<=r;i++)
        if(id[i+c]<=c) v[id[i+c]]=a[i][0];
    return a[0][0];
}
\end{lstlisting}


\section{Graph theory}
\subsection{Kruskal}
\begin{lstlisting}
struct EDGE
{
    int s,t,w;
    bool operator<(const EDGE &rhs) const
    {
        return w<rhs.w;
    }
};
EDGE edge[N_EDGE];
int m,sz,first[N_NODE],nxt[N_EDGE*2],tail[N_EDGE*2],len[N_EDGE*2],f[N_NODE];

int find(int u)
{
    return (f[u]==u ? u : (f[u]=find(f[u])));
}

void addedge(int u,int v,int l)
{
    sz++;
    tail[sz]=v;  len[sz]=l;
    nxt[sz]=first[u];  first[u]=sz;
}

int Kruskal()
{
    std::sort(edge+1,edge+m+1);
    int i,ans=0;
    for(i=1;i<=n;i++) f[i]=i;
    for(sz=0,i=1;i<=m;i++)
        if(find(edge[i].s)!=find(edge[i].t))
        {
            ans+=edge[i].w;
            f[find(edge[i].s)]=find(edge[i].t);
            addedge(edge[i].s,edge[i].t,edge[i].w);
            addedge(edge[i].t,edge[i].s,edge[i].w);
        }
    return ans;
}
\end{lstlisting}

\subsection{Dijkstra}
\begin{lstlisting}
#include <functional>

using namespace std;
typedef long long LL;
typedef pair<LL,LL> P;
LL sz,first[N_NODE],nxt[N_EDGE],tail[N_EDGE],len[N_EDGE],dist[N_NODE];

void addedge(LL u,LL v,LL w)
{
    sz++;
    tail[sz]=v;  len[sz]=w;
    nxt[sz]=first[u];  first[u]=sz;
}

void dijkstra(LL s)
{
    memset(dist,0x3f,sizeof(dist));
    priority_queue<P,vector<P>,greater<P>> q;
    q.push(P{0,s});
    LL u,v,e;
    while(!q.empty())
    {
        P t=q.top();  q.pop();
        if(t.first>=dist[t.second]) continue;
        u=t.second;
        dist[u]=t.first;
        for(e=first[u];e;e=nxt[e])
        {
            v=tail[e];
            if(dist[u]+len[e]<dist[v])
                q.push(P{dist[u]+len[e],v});
        }
    }
}

LL n,first[N_NODE],nxt[N_EDGE],tail[N_EDGE],len[N_EDGE],used[N_NODE],dist[N_NODE]

void dijkstra2(LL s)    // without priorty-queue
{
    memset(used,0,sizeof(used));
    memset(dist,0x3f,sizeof(dist));
    dist[s]=0;
    LL i,j,mind,mark,v,e;
    for(i=1;i<=n;i++)       // n is the number of nodes
    {
        for(mind=INF,j=1;j<=n;j++)
            if(!used[j] && dist[j]<mind)
            {
                mind=dist[j];
                mark=j;
            }
        used[mark]=1;
        for(e=first[mark];e;e=nxt[e])
        {
            v=tail[e];
            if(dist[mark]+len[e]<dist[v])
                dist[v]=dist[mark]+len[e];
        }
    }
}
\end{lstlisting}

\subsection{Bellman-Ford}
\begin{lstlisting}
int nv,ne,dist[N_NODE];

void bellman_ford(int s)
{
    int i,u,e;
    for(i=1;i<=nv;i++) dist[i]=INF;
    dist[s]=0;
    for(i=1;i<=nv-1;i++)
        for(u=1;u<=nv;u++)
            for(e=first[u];e;e=nxt[e])
                dist[tail[e]]=std::min(dist[tail[e]],dist[u]+len[e]);
}
\end{lstlisting}

\subsection{Topological sorting}
\begin{lstlisting}
int sz,first[N_NODE],nxt[N_EDGE],tail[N_EDGE],ideg[N_NODE],vis[N_NODE],topo[N_NODE],q[N_NODE];

void addedge(int u,int v)
{
    tail[++sz]=v;
    nxt[sz]=first[u];  first[u]=sz;
    ideg[v]++;
}

int toposort()
{
    int front=1,rear=0,i,u,v,e;
    memset(vis,0,sizeof(vis));
    for(i=1;i<=n;i++)
        if(!ideg[i])
        {
            q[++rear]=i;
            vis[i]=1;
        }
    i=0;
    while(front<=rear)
    {
        u=q[front++];
        topo[++i]=u;
        for(e=first[u];e;e=nxt[e])
        {
            v=tail[e];
            ideg[v]--;
            if(!ideg[v])
            {
                q[++rear]=v;
                vis[v]=1;
            }
        }
    }
    return i==n;
}
\end{lstlisting}

\subsection{Edmonds-Karp}
\begin{lstlisting}
int nv,ne,first[N_NODE],nxt[N_EDGE*2],tail[N_EDGE*2],rev[N_EDGE*2],from[N_EDGE*2],que[N_NODE];
double cap[N_EDGE*2],f[N_NODE];

void addedge(int u,int v,double c)
{
    ne++;
    tail[ne]=v;  cap[ne]=c;
    nxt[ne]=first[u];  first[u]=ne;
    ne++;
    tail[ne]=u;  cap[ne]=0;
    nxt[ne]=first[v];  first[v]=ne;
    rev[ne]=ne-1;  rev[ne-1]=ne;
}

double bfs(int s,int t)
{
    int front=1,rear=0,u,v,e;
    memset(f,0,sizeof(f));
    que[++rear]=s;
    f[s]=INF;
    while(front<=rear)
    {
        u=que[front++];
        if(u==t) break;
        for(e=first[u];e;e=nxt[e])
        {
            v=tail[e];
            if(f[v] || fabs(cap[e])<EPS) continue;
            que[++rear]=v;
            f[v]=std::min(f[u],cap[e]);
            from[v]=e;
        }
    }
    return f[t];
}

double flow(int s,int t)
{
    double res=0;
    int u;
    while(bfs(s,t))
    {
        for(u=t;u!=s;u=tail[rev[from[u]]])
        {
            cap[from[u]]-=f[t];
            cap[rev[from[u]]]+=f[t];
        }
        res+=f[t];
    }
    return res;
}
\end{lstlisting}

\subsection{Dinic}
\begin{lstlisting}
// pay attention to the initial value of cur[]!
int ne,nv,first[N_NODE],nxt[N_EDGE*2],tail[N_EDGE*2],cap[N_EDGE*2],rev[N_EDGE*2],cur[N_NODE],level[N_NODE],que[N_NODE];

void addedge(int u,int v,int c)
{
    ne++;
    tail[ne]=v;  cap[ne]=c;
    nxt[ne]=first[u];  first[u]=ne;
    ne++;
    tail[ne]=u;  cap[ne]=0;
    nxt[ne]=first[v];  first[v]=ne;
    rev[ne]=ne-1;  rev[ne-1]=ne;
}

void bfs(int s)
{
    memset(level,-1,sizeof(level));
    level[s]=0;
    int front=1,rear=1,u,v,e;
    que[1]=s;
    while(front<=rear)
    {
        u=que[front++];
        for(e=first[u];e;e=nxt[e])
        {
            v=tail[e];
            if(cap[e]==0 || level[v]>=0) continue;
            level[v]=level[u]+1;
            que[++rear]=v;
        }
    }
}

int dfs(int u,int t,int f)
{
    if(u==t) return f;
    int v,d,&e=cur[u];
    for(;e;e=nxt[e])
    {
        v=tail[e];
        if(cap[e]==0 || level[u]>=level[v]) continue;
        d=dfs(v,t,std::min(f,cap[e]));
        if(d>0)
        {
            cap[e]-=d;
            cap[rev[e]]+=d;
            return d;
        }
    }
    return 0;
}

int dinic(int s,int t)
{
    int ans=0,i,f;
    for(;;)
    {
        bfs(s);
        if(level[t]<0) break;
        for(i=1;i<=nv;i++)
            cur[i]=first[i];  // pay attention to the initial value of cur[]!
        while((f=dfs(s,t,INF))>0) ans+=f;
    }
    return ans;
}
\end{lstlisting}

\subsection{Min cost max flow (repeated Dijkstra)}
\begin{lstlisting}
using namespace std;
typedef pair<int,double> P;
int nv,ne,first[N_NODE],nxt[N_EDGE*2],tail[N_EDGE*2],cap[N_EDGE*2],rev[N_EDGE*2],from[N_EDGE*2],used[N_NODE];
LD len[N_EDGE*2],dist[N_NODE],h[N_NODE];

void addedge(int u,int v,int c,double l)
{
    ne++;
    tail[ne]=v;  cap[ne]=c;  len[ne]=l;
    nxt[ne]=first[u];  first[u]=ne;
    ne++;
    tail[ne]=u;  cap[ne]=0;  len[ne]=-l;
    nxt[ne]=first[v];  first[v]=ne;
    rev[ne]=ne-1;  rev[ne-1]=ne;
}

void bellman_ford(int s)
{
    int i,u,v,e;
    for(i=1;i<=nv;i++) dist[i]=INF;
    dist[s]=0;
    memset(from,0,sizeof(from));
    for(i=1;i<=nv-1;i++)
        for(u=1;u<=nv;u++)
            for(e=first[u];e;e=nxt[e])
            {
                if(!cap[e]) continue;
                v=tail[e];
                if(dist[u]+len[e]<dist[v])
                {
                    dist[v]=dist[u]+len[e];
                    from[v]=e;
                }
            }
}

void search(int s)
{
    memset(used,0,sizeof(used));
    int i,j,mark,v,e;
    double mind;
    for(i=1;i<=nv;i++) dist[i]=INF;
    dist[s]=0;
    memset(from,0,sizeof(from));
    for(i=1;i<=nv;i++)
    {
        for(mind=INF,j=1;j<=nv;j++)
            if(!used[j] && dist[j]<mind-EPS)
            {
                mind=dist[j];
                mark=j;
            } 
        used[mark]=1;
        for(e=first[mark];e;e=nxt[e])
        {
            v=tail[e];
            if(cap[e] && dist[mark]+len[e]+h[mark]-h[v]<dist[v]-EPS)
            {
                dist[v]=dist[mark]+len[e]+h[mark]-h[v];
                from[v]=e;
            }
        }
    }
}

P mcmf(int s,int t)
{
    double res=0;
    int i,u,flow=0,f;
    memset(h,0,sizeof(h));
    for(;;)
    {
        search(s);

        // When there are negative-weight edges initially, call bellman_ford(s) in the FIRST ITERATION.

        for(i=1;i<=nv;i++)
            h[i]+=dist[i];
        if(!from[t]) break;
        for(f=INF,u=t;u!=s;u=tail[rev[from[u]]])
            f=min(f,cap[from[u]]);
        for(u=t;u!=s;u=tail[rev[from[u]]])
        {
            cap[from[u]]-=f;
            cap[rev[from[u]]]+=f;
        }
        flow+=f;  res+=f*h[t];
    }
    return P(flow,res);
}
\end{lstlisting}

\subsection{Hungary}
\begin{lstlisting}
// edges are linked from left to right only!
void addedge(int u,int v)
{
    tail[++sz]=v;
    nxt[sz]=first[u];  first[u]=sz;
}

int hungary()
{
    int ret=0,i;
    memset(from,0,sizeof(from));
    for(i=1;i<=nl;i++)
    {
        memset(vis,0,sizeof(vis));
        if(match(i)) ret++;
    }
    return ret;
}

int match(int u)
{
    int v,e;
    for(e=first[u];e;e=nxt[e])
    {
        v=tail[e];
        if(vis[v]) continue;
        vis[v]=1;
        if(!from[v] || match(from[v]))
        {
            from[v]=u;
            return 1;
        }
    }
    return 0;
}
\end{lstlisting}

\subsection{SCC decomposition}
\begin{lstlisting}
int n,sz,first[N_NODE],nxt[N_EDGE],tail[N_EDGE],cnt,l,sz2,f2[N_NODE],n2[N_EDGE],t2[N_EDGE],scc[N_NODE],s[N_NODE],stk[N_NODE],dfn[N_NODE],low[N_NODE];

void addedge(int u,int v)
{
    tail[++sz]=v;
    nxt[sz]=first[u];  first[u]=sz;
}

void addedge2(int u,int v)
{
    t2[++sz2]=v;
    n2[sz2]=f2[u];  f2[u]=sz2;
}

void dfs(int u)
{
    scc[u]=-1;
    low[u]=dfn[u]=++dfn[0];
    stk[++l]=u;
    int v,e;
    for(e=first[u];e;e=nxt[e])
    {
        v=tail[e];
        if(scc[v]>0) continue;
        if(!scc[v]) dfs(v);
        low[u]=std::min(low[u],low[v]);
    }
    if(low[u]==dfn[u])
    {
        s[++cnt]=0;
        while(stk[l]!=u)
        {
            scc[stk[l--]]=cnt;
            s[cnt]++;
        }
        scc[stk[l--]]=cnt;
        s[cnt]++;
    }
}

void scc_dec()
{
    memset(scc,0,sizeof(scc));
    int i;
    for(cnt=0,l=0,i=1;i<=n;i++)
        if(!scc[i]) dfs(i);
    memset(f2,0,sizeof(f2));
    int u,e;
    for(sz2=0,u=1;u<=n;u++)
        for(e=first[u];e;e=nxt[e])
            if(scc[u]!=scc[tail[e]])
                addedge2(scc[u],scc[tail[e]]);
}
\end{lstlisting}

\subsection{2-SAT}
\begin{lstlisting}
int solve()
{
    memset(dep,0,sizeof(dep));
    for(dfn[0]=scc[0]=cc[0]=0,lv=0,i=1;i<=m*2;i++)
        if(!dep[i])
        {
            ++cc[0];
            dep[i]=1;
            dfs(i);
        }
    for(i=1;i<=m;i++)
        if(scc[i]==scc[i+m]) return 0;
        else if(cc[i]<cc[i+m] || cc[i]==cc[i+m] && dep[i]>dep[i+m]) chosen[i]=1;
        else chosen[i]=0;
    return 1;
}

void dfs(int u)
{
    low[u]=dfn[u]=++dfn[0];
    scc[u]=-1;
    cc[u]=cc[0];
    stk[++lv]=u;
    int v,e;
    for(e=first[u];e;e=nxt[e])
    {
        v=tail[e];
        if(!dep[v])
        {
            dep[v]=dep[u]+1;
            dfs(v);
            low[u]=min(low[u],low[v]);
        }
        else if(scc[v]==-1)
            low[u]=min(low[u],low[v]);
    }
    if(low[u]==dfn[u])
    {
        scc[0]++;
        while(stk[lv]!=u)
            scc[stk[lv--]]=scc[0];
        scc[stk[lv--]]=scc[0];
    }
}

\end{lstlisting}

\subsection{LCA (doubling)}
\begin{lstlisting}
LL sz,first[N_NODE],nxt[N_NODE*2],tail[N_NODE*2],len[N_NODE*2],p[N_NODE][LOG_N_NODE],g[N_NODE][LOG_N_NODE],dep[N_NODE];

void addedge(LL u,LL v,LL w)
{
    sz++;
    tail[sz]=v;  len[sz]=w;
    nxt[sz]=first[u];  first[u]=sz;
}

void dfs(LL u)
{
    LL v,e,i;
    for(i=1;i<=LOG_N;i++)
    {
        p[u][i]=p[p[u][i-1]][i-1];
        g[u][i]=max(g[u][i-1],g[p[u][i-1]][i-1]);
    }
    for(e=first[u];e;e=nxt[e])
    {
        v=tail[e];
        if(dep[v]) continue;
        p[v][0]=u;  g[v][0]=len[e];
        dep[v]=dep[u]+1;
        dfs(v);
    }
}

LL query(LL u,LL v)
{
    if(dep[u]>dep[v]) swap(u,v);
    LL i,ans=0;
    for(i=LOG_N;i>=0;i--)
        if(dep[v]-dep[u]>=(1<<i))
        {
            ans=max(ans,g[v][i]);
            v=p[v][i];
        }
    if(u==v) return ans;
          // return u;
    for(i=LOG_N;i>=0;i--)
        if(p[u][i]!=p[v][i])
        {
            ans=max(ans,max(g[u][i],g[v][i]));
            u=p[u][i];  v=p[v][i];
        }
    return max(ans,max(g[u][0],g[v][0]));
    // return p[u][0];
}
\end{lstlisting}

\subsection{LCA (RMQ)}
\begin{lstlisting}
int n,sz,first[N_NODE],nxt[N_NODE*2],tail[N_NODE*2],dep[N_NODE],li[N_NODE],eul[N_NODE*2],b,minv[LOG_N_NODE][N_NODE*2];

void addedge(int u,int v)
{
    tail[++sz]=v;
    nxt[sz]=first[u];  first[u]=sz;
}

void dfs(int u)
{
    eul[++eul[0]]=u;    // initialize eul[0]
    li[u]=eul[0];   // initialize li[] to 0
    int v,e;
    for(e=first[u];e;e=nxt[e])
    {
        v=tail[e];
        if(li[v]) continue;
        dep[v]=dep[u]+1;
        dfs(v);
        eul[++eul[0]]=u;
    }
}

void rmq()
{
    b=sizeof(unsigned int)*8-__builtin_clz(n*2-1)-1;    // int __builtin_clz(unsigned int x)
    int i,j;
    for(i=1;i<=n*2-1;i++)
        minv[0][i]=eul[i];
    for(i=1;i<=b;i++)
        for(j=1;j<=n*2-1;j++)
            if(j+(1<<(i-1))>n*2-1 || dep[minv[i-1][j]]<dep[minv[i-1][j+(1<<(i-1))]])
                minv[i][j]=minv[i-1][j];
            else minv[i][j]=minv[i-1][j+(1<<(i-1))];
}

int query(int l,int r)
{
    int w=sizeof(unsigned int)*8-__builtin_clz(r-l+1)-1;
    if(dep[minv[w][l]]<dep[minv[w][r-(1<<w)+1]])
        return minv[w][l];
    else return minv[w][r-(1<<w)+1];
}

int lca(int u,int v)
{
    return query(std::min(li[u],li[v]),std::max(li[u],li[v]));
}
\end{lstlisting}

\subsection{LCA (Tarjan)}
\begin{lstlisting}
#include <cstdio>
#include <cstring>
#include <algorithm>

using namespace std;
const int INF=1000000000;
struct PLAN
{
    int s,t,time,idx;
    bool operator<(const PLAN &rhs) const
    {
        return time<rhs.time;
    }
}
plan[N_QRY];
int n,m,nq,firstq[N_NODE],nxtq[N_QRY*2],to[N_QRY*2],idx[N_QRY*2],lca[N_QRY],sz,first[N_NODE],nxt[N_NODE*2],tail[N_NODE*2],len[N_NODE*2],vis[N_NODE],f[N_NODE];

void addedge(int u,int v,int l)
{
    sz++;
    tail[sz]=v;  len[sz]=l;
    nxt[sz]=first[u];  first[u]=sz;
}

void addqry(int u,int v,int i)
{
    nq++;
    to[nq]=v;  idx[nq]=i;
    nxtq[nq]=firstq[u];  firstq[u]=nq;
}

int find(int u)
{
    return (f[u]==u ? u : (f[u]=find(f[u])));
}

void get_lca(int u)
{
    vis[u]=1;
    int q,v,e;
    for(q=firstq[u];q;q=nxtq[q])
    {
        v=to[q];
        if(!vis[v]) continue;
        lca[idx[q]]=find(v);
    }
    for(e=first[u];e;e=nxt[e])
    {
        v=tail[e];
        if(vis[v]) continue;
        get_lca(v);
        f[v]=u;
    }
}

int main()
{
    scanf("%d%d",&n,&m);
    memset(first,0,sizeof(first));
    int i,u,v,l;
    for(sz=0,i=1;i<=n-1;i++)
    {
        scanf("%d%d%d",&u,&v,&l);
        addedge(u,v,l);  addedge(v,u,l);
    }
    memset(firstq,0,sizeof(firstq));
    for(nq=0,i=1;i<=m;i++)
    {
        scanf("%d%d",&u,&v);
        plan[i].s=u;  plan[i].t=v;
        plan[i].idx=i;
        addqry(u,v,i);  addqry(v,u,i);
    }
    for(i=1;i<=n;i++) f[i]=i;
    memset(vis,0,sizeof(vis));
    get_lca(1);  // 1 is the root
    return 0;
}
\end{lstlisting}

\subsection{Diameter of tree}
\begin{lstlisting}
int n,sz,first[N_NODE],nxt[N_NODE*2],tail[N_NODE*2],dep[N_NODE],par[N_NODE],diam[N_NODE];

void dfs(int u)
{
    int v,e;
    for(e=first[u];e;e=nxt[e])
    {
        v=tail[e];
        if(dep[v]) continue;
        par[v]=u;
        dep[v]=dep[u]+1;
        dfs(v);
    }
}

int diameter(int &rt)   // return length of tree's diameter
{
    int i;
    memset(dep,0,sizeof(dep));
    dep[1]=1;
    dfs(1);
    for(rt=0,i=1;i<=n;i++)
        if(dep[i]>dep[rt]) rt=i;
    memset(dep,0,sizeof(dep));
    dep[rt]=1;
    dfs(rt);
    int t=0;
    for(i=1;i<=n;i++)
        if(dep[i]>dep[t]) t=i;
    int u,len=0;
    for(u=t;u!=rt;u=par[u])
        diam[++len]=u;
    diam[++len]=rt;
    return len;
}
\end{lstlisting}

\subsection{Maximum density subgraph}
\begin{lstlisting}
#include <cstdio>
#include <cstring>
#include <cmath>
#include <algorithm>

const double INF=1e5,EPS=1e-18,LIM=1e-5;
int n,m,cv,ce,first[2000],nxt[8000],tail[8000],rev[8000],level[2000],cur[2000],que[2000],ans[110];
double cap[8000],cap0[8000];
void addedge(int u,int v,double c);
double dinic(int s,int t);
void bfs(int s);
double dfs(int u,int t,double f);

int main()
{
    freopen("life.in","r",stdin);
    freopen("life.out","w",stdout);
    scanf("%d%d",&n,&m);
    memset(first,0,sizeof(first));
    int i,u,v;
    for(ce=0,i=1;i<=m;i++)
    {
        scanf("%d%d",&u,&v);
        addedge(i,u+m,INF);  addedge(i,v+m,INF);
    }
    int s=n+m+1,t=n+m+2;
    for(i=1;i<=m;i++) addedge(s,i,1);
    for(i=1;i<=n;i++) addedge(i+m,t,0);
    for(i=1;i<=ce;i++) cap0[i]=cap[i];
    cv=n+m+2;
    double l=0,r=m,mid;
    while(r-l>LIM)
    {
        mid=(l+r)/2;
        for(i=1;i<=ce;i++)
            if(tail[i]==t) cap[i]=mid;
            else cap[i]=cap0[i];
        if(m-dinic(s,t)>EPS)
        {
            l=mid;
            for(ans[0]=0,i=1;i<=n;i++)
                if(level[i+m]>=0)
                    ans[++ans[0]]=i;
        }
        else r=mid;
    }
    std::sort(ans+1,ans+ans[0]+1);
    if(!ans[0]) ans[ans[0]=1]=1;
    for(i=0;i<=ans[0];i++)
        printf("%d\n",ans[i]);
    return 0;
}

void addedge(int u,int v,double c)
{
    ce++;
    tail[ce]=v;  cap[ce]=c;
    nxt[ce]=first[u];  first[u]=ce;
    ce++;
    tail[ce]=u;  cap[ce]=0;
    nxt[ce]=first[v];  first[v]=ce;
    rev[ce]=ce-1;  rev[ce-1]=ce;
}

double dinic(int s,int t)
{
    double ans=0,f;
    int i;
    for(;;)
    {
        bfs(s);
        if(level[t]<0) break;
        for(i=1;i<=cv;i++)
            cur[i]=first[i];
        while((f=dfs(s,t,INF))>0) ans+=f;
    }
    return ans;
}

void bfs(int s)
{
    memset(level,-1,sizeof(level));
    level[s]=0;
    int front=1,rear=1,u,v,e;
    que[1]=s;
    while(front<=rear)
    {
        u=que[front++];
        for(e=first[u];e;e=nxt[e])
        {
            v=tail[e];
            if(fabs(cap[e])<EPS || level[v]>=0) continue;
            level[v]=level[u]+1;
            que[++rear]=v;
        }
    }
}

double dfs(int u,int t,double f)
{
    if(u==t) return f;
    int v,&e=cur[u];
    double d;
    for(;e;e=nxt[e])
    {
        v=tail[e];
        if(fabs(cap[e])<EPS || level[u]>=level[v]) continue;
        d=dfs(v,t,std::min(f,cap[e]));
        if(d>0)
        {
            cap[e]-=d;
            cap[rev[e]]+=d;
            return d;
        }
    }
    return 0;
}
\end{lstlisting}


\section{Data structures}
\subsection{Discretization}
\begin{lstlisting}
int n,l[N_SEG],r[N_SEG],l2[N_SEG],r2[N_SEG],x[N_SEG*2];

void discrete()
{
    int i,j;
    for(i=1;i<=n;i++)
        x[2*i-1]=l[i],x[2*i]=r[i];
    std::sort(x+1,x+2*n+1);
    int *arr[2]={l,r},*arrn[2]={l2,r2},lb,ub,mid;
    for(i=1;i<=n;i++)
        for(j=0;j<=1;j++)
        {
            lb=1;  ub=n*2;
            while(lb<ub)
            {
                mid=(lb+ub)/2;
                if(arr[j][i]<=x[mid]) ub=mid;
                else lb=mid+1;
            }
            arrn[j][i]=lb;
        }
}
\end{lstlisting}

\subsection{Segment tree (add)}
\begin{lstlisting}
void update(int p,int q,LL val,int k,int l,int r)
{
    if(p<=l && q>=r)
    {
        add[k]+=val;
        maintain(k,l,r);
        return;
    }
    int mid=(l+r)/2;
    if(p<=mid) update(p,q,val,k*2,l,mid);
    if(q>mid) update(p,q,val,k*2+1,mid+1,r);
    maintain(k,l,r);
}

void query(int p,int q,int k,int l,int r,LL s,LL &res)
{
    if(p<=l && q>=r)
    {
        res+=sum[k]+s*(r-l+1);
        return;
    }
    int mid=(l+r)/2;
    if(p<=mid) query(p,q,k*2,l,mid,s+add[k],res);
    if(q>mid) query(p,q,k*2+1,mid+1,r,s+add[k],res);
}

void maintain(int k,int l,int r)
{
    if(l==r) sum[k]=add[k];
    else sum[k]=sum[k*2]+sum[k*2+1]+add[k]*(r-l+1);
}

\end{lstlisting}

\subsection{Segment tree (add, set)}
\begin{lstlisting}
#include <cstdio>
#include <cstring>
#include <algorithm>

using namespace std;
typedef long long LL;
const LL INF=100000000000000;
LL n,len,q,a[1000010],tag[2][2000010],sum[2000010],mx[2000010],mn[2000010],ans[3];
void build(LL k,LL l,LL r);
void update(LL op,LL p,LL q,LL val,LL k,LL l,LL r);
void query(LL p,LL q,LL k,LL l,LL r,LL add,LL &res_s,LL &res_a,LL &res_i);
void maintain(LL k,LL l,LL r);
void pushdown(LL k,LL l,LL r);

int main()
{
    freopen("segment.in","r",stdin);
    freopen("segment.out","w",stdout);
    scanf("%lld",&n);
    memset(a,0,sizeof(a));
    LL i;
    for(i=1;i<=n;i++)
        scanf("%lld",&a[i]);
    for(len=1;len<n;len*=2);
    build(1,1,len);
    scanf("%lld",&q);
    LL op,l,r,x;
    while(q--)
    {
        scanf("%lld",&op);
        if(op==1)
        {
            scanf("%lld%lld%lld",&l,&r,&x);
            update(1,l,r,x,1,1,len);
        }
        else if(op==2)
        {
            scanf("%lld%lld%lld",&l,&r,&x);
            update(0,l,r,x,1,1,len);
        }
        else
        {
            scanf("%lld%lld",&l,&r);
            ans[0]=0,ans[1]=-INF,ans[2]=INF;
            query(l,r,1,1,len,0,ans[0],ans[1],ans[2]);
            printf("%lld %lld %lld\n",ans[0],ans[1],ans[2]);
        }
    }
    return 0;
}

void build(LL k,LL l,LL r)
{
    if(l==r)
    {
        tag[0][k]=a[l];  tag[1][k]=0;
        maintain(k,l,r);
        return;
    }
    LL mid=(l+r)/2;
    build(k*2,l,mid);  build(k*2+1,mid+1,r);
    tag[0][k]=INF;  tag[1][k]=0;
    maintain(k,l,r);
}

void update(LL op,LL p,LL q,LL val,LL k,LL l,LL r)
{
    if(p<=l && q>=r)
    {
        if(op) tag[1][k]+=val;
        else tag[0][k]=val,tag[1][k]=0;
        maintain(k,l,r);
        return;
    }
    pushdown(k,l,r);
    LL mid=(l+r)/2;
    if(p<=mid) update(op,p,q,val,k*2,l,mid);
    if(q>mid) update(op,p,q,val,k*2+1,mid+1,r);
    maintain(k,l,r);
}

void query(LL p,LL q,LL k,LL l,LL r,LL add,LL &res_s,LL &res_a,LL &res_i)
{
    if(p<=l && q>=r)
    {
        res_s+=sum[k]+add*(r-l+1);
        res_a=max(res_a,mx[k]+add);  res_i=min(res_i,mn[k]+add);
        return;
    }
    else if(tag[0][k]!=INF)
    {
        res_s+=(tag[0][k]+tag[1][k]+add)*(min(q,r)-max(p,l)+1);
        res_a=max(res_a,tag[0][k]+tag[1][k]+add);
        res_i=min(res_i,tag[0][k]+tag[1][k]+add);
        return;
    }
    LL mid=(l+r)/2;
    if(p<=mid) query(p,q,k*2,l,mid,add+tag[1][k],res_s,res_a,res_i);
    if(q>mid) query(p,q,k*2+1,mid+1,r,add+tag[1][k],res_s,res_a,res_i);
}

void maintain(LL k,LL l,LL r)
{
    if(tag[0][k]!=INF)
    {
        sum[k]=(tag[0][k]+tag[1][k])*(r-l+1);
        mx[k]=mn[k]=tag[0][k]+tag[1][k];
    }
    else
    {
        sum[k]=sum[k*2]+sum[k*2+1]+tag[1][k]*(r-l+1);
        mx[k]=max(mx[k*2],mx[k*2+1])+tag[1][k];
        mn[k]=min(mn[k*2],mn[k*2+1])+tag[1][k];
    }
}

void pushdown(LL k,LL l,LL r)
{
    if(tag[0][k]!=INF)
    {
        tag[0][k*2]=tag[0][k*2+1]=tag[0][k];
        tag[1][k*2]=tag[1][k*2+1]=tag[1][k];
    }
    else
    {
        tag[1][k*2]+=tag[1][k];  tag[1][k*2+1]+=tag[1][k];
    }
    LL mid=(l+r)/2;
    maintain(k*2,l,mid);  maintain(k*2+1,mid+1,r);
    tag[0][k]=INF;  tag[1][k]=0;
    maintain(k,l,r);
}
\end{lstlisting}

\subsection{Segment tree (add, multiply)}
\begin{lstlisting}
int n,m,son[100010],bro[100010],cc,s[100010],par[100010],idx[100010],belong[100010],top[100010],mulv[400010],addv[400010],sum[400010];

void maintain(int k,int l,int r)
{
    if(l==r) sum[k]=addv[k];
    else sum[k]=(sum[k*2]+sum[k*2+1])*mulv[k]+addv[k]*(r-l+1);
}

void pushdown(int k,int l,int r)
{
    int mid=(l+r)/2;
    mulv[k*2]*=mulv[k];  addv[k*2]=addv[k*2]*mulv[k]+addv[k];
    maintain(k*2,l,mid);
    mulv[k*2+1]*=mulv[k];  addv[k*2+1]=addv[k*2+1]*mulv[k]+addv[k];
    maintain(k*2+1,mid+1,r);
    mulv[k]=1;  addv[k]=0;
    maintain(k,l,r);
}

void build(int k,int l,int r)
{
    mulv[k]=1;  addv[k]=0;
    if(l<r)
    {
        int mid=(l+r)/2;
        build(k*2,l,mid);  build(k*2+1,mid+1,r);
    }
    maintain(k,l,r);
}

void update(int p,int q,int val,int o,int k,int l,int r)    // o=0: multiply; o=1: add
{
    if(p<=l && q>=r)
    {
        if(!o) mulv[k]*=val,addv[k]*=val;
        else addv[k]+=val;
        maintain(k,l,r);
        return;
    }
    pushdown(k,l,r);
    int mid=(l+r)/2;
    if(p<=mid) update(p,q,val,o,k*2,l,mid);
    if(q>mid) update(p,q,val,o,k*2+1,mid+1,r);
    maintain(k,l,r);
}

int query(int p,int q,int k,int l,int r,int mul,int add)
{
    if(p<=l && q>=r)
        return sum[k]*mul+add*(r-l+1);
    int mid=(l+r)/2,res=0;
    add=mul*addv[k]+add;  mul*=mulv[k];
    if(p<=mid) res+=query(p,q,k*2,l,mid,mul,add);
    if(q>mid) res+=query(p,q,k*2+1,mid+1,r,mul,add);
    return res;
}
\end{lstlisting}

\subsection{Segment tree (sum of sums)}
\begin{lstlisting}
void build(int k,int l,int r)
{
    if(l==r)
    {
        ls[k]=rs[k]=ss[k]=sum[k]=a[l];
        return;
    }
    int mid=(l+r)/2;
    build(k*2,l,mid);  build(k*2+1,mid+1,r);
    sum[k]=(sum[k*2]+sum[k*2+1])%M;
    ls[k]=(ls[k*2]+ls[k*2+1]+sum[k*2]*(r-mid)%M)%M;
    rs[k]=(rs[k*2]+rs[k*2+1]+sum[k*2+1]*(mid-l+1)%M)%M;
    ss[k]=(ss[k*2]+ss[k*2+1]+rs[k*2]*(r-mid)%M+ls[k*2+1]*(mid-l+1)%M)%M;
}

LL query(int p,int q,int k,int l,int r)
{
    if(p<=l && q>=r)
        return (ss[k]+sum[k]*(l-p)%M*(q-r)%M+ls[k]*(l-p)%M+rs[k]*(q-r)%M)%M;
    int mid=(l+r)/2;
    LL res=0;
    if(p<=mid) res=(res+query(p,q,k*2,l,mid))%M;
    if(q>mid) res=(res+query(p,q,k*2+1,mid+1,r))%M;
    return res;
}
\end{lstlisting}

\subsection{Heap}
\begin{lstlisting}
struct HEAPNODE
{
    int val,idx;
};

void up(int u)
{
    if(u>1 && heap[u].val>heap[u/2].val)
    {
        std::swap(heap[u],heap[u/2]);
        up(u/2);
    }
}

void down(int u)
{
    if(u*2>h) return;
    int v=u*2;
    if(v+1<=h && heap[v+1].val>heap[v].val) v++;
    if(heap[v].val>heap[u].val)
    {
        std::swap(heap[u],heap[v]);
        down(v);
    }
}

void push(int val,int idx)
{
    h++;
    heap[h].val=val;  heap[h].idx=idx;
    up(h);
}

HEAPNODE pop()
{
    HEAPNODE res=heap[1];
    std::swap(heap[1],heap[h--]);
    down(1);
    return res;
}
\end{lstlisting}

\subsection{RMQ}
\begin{lstlisting}
int n,b,arr[SIZE],minv[LOG_SIZE][SIZE*2];

void rmq()
{
    b=sizeof(unsigned int)*8-__builtin_clz(n)-1;        // int __builtin_clz(unsigned int x)
    memset(minv,0x3f,sizeof(minv));
    int i,j;
    for(i=1;i<=n;i++)
        minv[0][i]=arr[i];
    for(i=1;i<=b;i++)
        for(j=1;j<=n;j++)
            minv[i][j]=std::min(minv[i-1][j],minv[i-1][j+(1<<(i-1))]);
}

int query(int l,int r)
{
    int w=sizeof(unsigned int)*8-__builtin_clz(r-l+1)-1;
    return std::min(minv[w][l],minv[w][r-(1<<w)+1]);
}
\end{lstlisting}

\subsection{Weighted union find}
\begin{lstlisting}
int n,k,f[N_NODE],d[N_NODE];  // initialize f[i] to i, d[i] to 0

int find(int u)
{
    if(f[u]==u) return u;
    int t=find(f[u]);
    d[u]=(d[u]+d[f[u]])%MOD;
    f[u]=t;
    return t;
}

int dist(int u)
{
    find(u);
    return d[u];
}

int link(int u,int v,int l)     // link u to v (l is the distance from u to v)
{
    int fu=find(u),fv=find(v);
    if(fu==fv) return 0;       // return 0 if u and v have been in the same set
    f[fu]=fv;
    d[fu]=(l+d[v]-d[u]+MOD)%MOD;
    return 1;
}
\end{lstlisting}

\subsection{Treap}
\begin{lstlisting}
const unsigned SEED=19260817;
int sz,ls[N_NODE],rs[N_NODE],par[N_NODE],key[N_NODE],val[N_NODE],hk[N_NODE],cnt[N_NODE];

void maintain(int u)
{
    cnt[u]=cnt[ls[u]]+cnt[rs[u]]+1;
}

int lrot(int u)
{
    int v=rs[u];
    rs[u]=ls[v];  ls[v]=u;
    par[v]=par[u];  par[u]=v;  par[rs[u]]=u;
    maintain(u);  maintain(v);
    return v;
}

int rrot(int u)
{
    int v=ls[u];
    ls[u]=rs[v];  rs[v]=u;
    par[v]=par[u];  par[u]=v;  par[ls[u]]=u;
    maintain(u);  maintain(v);
    return v;
}

int insert(int u,int p,int kk,int vv)
{
    if(!u)
    {
        sz++;
        par[sz]=p;  ls[sz]=rs[sz]=0;
        key[sz]=kk;  val[sz]=vv;
        hk[sz]=rand()+1;
        cnt[sz]=1;
        return sz;
    }
    if(kk<key[u])
        ls[u]=insert(ls[u],u,kk,vv);
    else rs[u]=insert(rs[u],u,kk,vv);
    maintain(u);
    if(hk[ls[u]]>hk[u]) u=rrot(u);
    else if(hk[rs[u]]>hk[u]) u=lrot(u);
    return u;
}

int erase(int u,int targ)
{
    if(u==targ)
    {
        if(!ls[u] && !rs[u])
        {
            par[u]=0;
            return 0;
        }
        if(hk[ls[u]]>hk[rs[u]])
        {
            u=rrot(u);
            rs[u]=erase(rs[u],targ);
        }
        else
        {
            u=lrot(u);
            ls[u]=erase(ls[u],targ);
        }
        maintain(u);
        return u;
    }
    if(key[targ]<key[u])
        ls[u]=erase(ls[u],targ);
    else rs[u]=erase(rs[u],targ);
    maintain(u);
    return u;
}

int newnode(int kk,int vv)
{
    sz++;
    par[sz]=ls[sz]=rs[sz]=0;
    key[sz]=kk;  val[sz]=vv;
    hk[sz]=rand()+1;
    cnt[sz]=1;
    return sz;
}

int insert_node(int u,int p,int v)
{
    if(!u)
    {
        par[v]=p;
        return v;
    }
    if(key[v]<key[u])
        ls[u]=insert_node(ls[u],u,v);
    else rs[u]=insert_node(rs[u],u,v);
    maintain(u);
    if(hk[ls[u]]>hk[u]) u=rrot(u);
    else if(hk[rs[u]]>hk[u]) u=lrot(u);
    return u;
}

int find(int u,int kk)
{
    if(!u) return 0;
    else if(kk==key[u]) return u;
    else if(kk<key[u])
        return find(ls[u],kk);
    else return find(rs[u],kk);
}

int pred(int u)
{
    if(ls[u])
    {
        int v=ls[u];
        while(rs[v]) v=rs[v];
        return v;
    }
    int p=u,q=par[u];
    while(q && p==ls[q])
        p=q,q=par[q];
    return q;
}

int succ(int u)
{
    if(rs[u])
    {
        int v=rs[u];
        while(ls[v]) v=ls[v];
        return v;
    }
    int p=u,q=par[u];
    while(q && p==rs[q])
        p=q,q=par[q];
    return q;
}

int rank(int u,int kk)
{
    if(!u) return 0;
    else if(kk==key[u]) return cnt[ls[u]];
    else if(kk<key[u])
        return rank(ls[u],kk);
    else return rank(rs[u],kk)+cnt[ls[u]]+1;
}

int select(int u,int k)
{
    int t=cnt[ls[u]];
    if(k==t+1) return u;
    else if(k<t+1)
        return select(ls[u],k);
    else return select(rs[u],k-cnt[ls[u]]-1);
}

void merge(int &dest,int &src)
{
    int t;
    while(src)
    {
        t=src;
        src=erase(src,src);
        dest=insert_node(dest,0,t);
    }
}
\end{lstlisting}

\subsection{Cartesian tree}
\begin{lstlisting}
int n,ls[N_NODE],rs[N_NODE],stk[N_NODE];

int carttree(int *a)
{
    int i,m=0,rt=0;
    for(i=1;i<=n;i++)
    {
        while(m && a[i]>a[stk[m]]) m--;         // the largest value is the root
        ls[i]=m ? rs[stk[m]] : rt;
        rs[i]=0;
        if(m) rs[stk[m]]=i;
        else rt=i;
        stk[++m]=i;
    }
    return rt;
}
\end{lstlisting}

\subsection{Expression tree}
\begin{lstlisting}
int build_expr(char *str)    // return root of expression tree
{
    int numn=0,opn=0,l=strlen(str),i;
    for(i=0;i<l;i++)
    {
        if(str[i]>='0' && str[i]<='9') nums[++numn]='0'-str[i];    // leaves(operands) have negative tree-node indices
        else if(str[i]=='(' || str[i]=='?') ops[++opn]=str[i];
        else
        {
            while(opn && ops[opn]!='(')
            {
                sz++;
                rs[sz]=nums[numn--];  ls[sz]=nums[numn--];
                s[sz]=(ls[sz]>0 ? s[ls[sz]] : 0)+(rs[sz]>0 ? s[rs[sz]] : 0)+1;
                nums[++numn]=sz;
                opn--;
            }
            if(opn) opn--;
        }
    }
    return nums[1];
}
\end{lstlisting}

\subsection{Persistent segment tree}
\begin{lstlisting}
// updates for single positions, and querys for prefix sums (similar to BIT)

LL n,a[100010],sz,ls[3000000],rs[3000000],sum[3000000],rt[100010];

LL build(LL l,LL r)
{
    LL u=++sz;      // sz needs to be initialized
    sum[u]=0;
    if(l<r)
    {
        LL m=(l+r)/2;
        ls[u]=build(l,m);  rs[u]=build(m+1,r);
    }
    return u;
}

LL update(LL pos,LL val,LL u,LL l,LL r)
{
    if(l==r)
    {
        sum[++sz]=sum[u]+val;
        return sz;
    }
    LL v=++sz,m=(l+r)/2;
    if(pos<=m)
    {
        ls[v]=update(pos,val,ls[u],l,m);
        rs[v]=rs[u];
    }
    else
    {
        ls[v]=ls[u];
        rs[v]=update(pos,val,rs[u],m+1,r);
    }
    sum[v]=sum[ls[v]]+sum[rs[v]];
    return v;
}

LL query(LL pos,LL u,LL l,LL r)
{
    if(pos<l) return 0;
    else if(pos==r) return sum[u];
    LL m=(l+r)/2;
    if(pos>m) return sum[ls[u]]+query(pos,rs[u],m+1,r);
    else return query(pos,ls[u],l,m);
}

int main()
{
    scanf("%lld",&n);
    LL i;
    for(i=1;i<=n;i++)
        scanf("%lld",&a[i]);
    sz=0;
    rt[0]=build(1,n);
    for(i=1;i<=n;i++)
        rt[i]=update(a[i],1,rt[i-1],1,n);
    return 0;
}
\end{lstlisting}

\subsection{Chain decomposition}
\begin{lstlisting}
int cc,s[N_NODE],par[N_NODE],idx[N_NODE],belong[N_NODE],top[N_NODE];

void dfs(int u)
{
    int v,e;
    for(s[u]=1,e=first[u];e;e=nxt[e])
    {
        v=tail[e];
        if(v==par[u]) continue;         // par[root] should be initialized
        par[v]=u;
        dfs(v);
        s[u]+=s[v];
    }
}

void split(int u)
{
    idx[u]=++idx[0];    // idx[0] should be initialized
    int v,e,v0=0;
    for(e=first[u];e;e=nxt[e])
    {
        v=tail[e];
        if(par[v]!=u) continue;
        if(s[v]>s[v0]) v0=v;
    }
    if(!v0)
    {
        belong[u]=++cc;     // cc should be initialized
        top[cc]=u;
        // ridx[u]=idx[0];
        return;
    }
    split(v0);
    belong[u]=belong[v0];
    top[belong[u]]=u;
    for(e=first[u];e;e=nxt[e])
    {
        v=tail[e];
        if(par[v]!=u || v==v0) continue;
        split(v);
    }
    // ridx[u]=idx[0];
}

int lca(int u,int v)
{
    while(belong[u]!=belong[v])
    {
        if(idx[top[belong[u]]]>idx[top[belong[v]]]) std::swap(u,v);
        v=par[top[belong[v]]];
    }
    return idx[u]<idx[v] ? u : v;
}

void update2(int u,int v,int val)
{
    while(belong[u]!=belong[v])
    {
        if(idx[top[belong[u]]]>idx[top[belong[v]]]) std::swap(u,v);
        // update(idx[top[belong[v]]],idx[v],val,1,1,n);
        v=par[top[belong[v]]];
    }
    if(idx[u]>idx[v]) std::swap(u,v);
    // update(idx[u],idx[v],val,1,1,n);
}

int query2(int u,int v)
{
    int res=0;
    while(belong[u]!=belong[v])
    {
        if(idx[top[belong[u]]]>idx[top[belong[v]]]) std::swap(u,v);
        // res+=query(idx[top[belong[v]]],idx[v],1,1,n);
        v=par[top[belong[v]]];
    }
    if(idx[u]>idx[v]) std::swap(u,v);
    // res+=query(idx[u],idx[v],1,1,n);
    return res;
}
\end{lstlisting}


\section{String}
\subsection{KMP}
\begin{lstlisting}
int n,m,fail[LEN_PATTERN],match[LEN],trans[LEN_PATTERN][ALPHABET];

void kmp(int *s1,int *s2)
{
    int i,j;
    for(fail[0]=fail[1]=0,i=2;i<=m;i++)
    {
        for(j=fail[i-1];j;j=fail[j])
            if(s2[j+1]==s2[i])
            {
                fail[i]=j+1;
                break;
            }
        if(!j) fail[i]=(s2[1]==s2[i] ? 1 : 0);
    }
    memset(trans[0],0,sizeof(trans[0]));
    trans[0][s2[1]]=1;
    for(i=1;i<=m;i++)
        for(j=1;j<=ALPHABET;j++)
            trans[i][j]=(i<m && s2[i+1]==j) ? i+1 : trans[fail[i]][j];
    for(match[0]=0,i=1;i<=n;i++)
    {
        for(j=match[i-1];j;j=fail[j])
            if(j+1<=m && s2[j+1]==s1[i])
            {
                match[i]=j+1;
                break;
            }
        if(!j) match[i]=(s2[1]==s1[i] ? 1 : 0);

        match[i]=trans[match[i-1]][s1[i]];      // with array trans[][]
    }
}
\end{lstlisting}

\subsection{Aho-Corasick automaton}
\begin{lstlisting}
char str[N_STRING][MAX_LEN];
int n,sz,son[N_NODE][ALPHABET],par[N_NODE],alp[N_NODE],fail[N_NODE],que[N_NODE],term[N_NODE],mat[LENGTH],trans[N_NODE][ALPHABET];

void build_trie()
{
    int i,j,u;
    memset(son[1],0,sizeof(son[1]));
    term[1]=0;
    for(sz=1,i=1;i<=n;i++)
    {
        for(u=1,j=0;str[i][j];j++)
        {
            if(!son[u][str[i][j]-'a'])      // lowercase only
            {
                son[u][str[i][j]-'a']=++sz;
                memset(son[sz],0,sizeof(son[sz]));
                par[sz]=u;
                alp[sz]=str[i][j]-'a';
                term[sz]=0;
            }
            u=son[u][str[i][j]-'a'];
        }
        term[u]=1;
    }
}

void build_ac()
{
    int front=1,rear=1,u,v,j;
    que[1]=1;
    while(front<=rear)
    {
        u=que[front++];
        if(u==1 || par[u]==1) fail[u]=1;
        else
        {
            for(v=fail[par[u]];v>1;v=fail[v])
                if(son[v][alp[u]])
                {
                    fail[u]=son[v][alp[u]];
                    break;
                }
            if(v==1) fail[u]=(son[1][alp[u]] ? son[1][alp[u]] : 1);
        }
        term[u]|=term[fail[u]];
        for(j=0;j<26;j++)
            if(u==1) trans[u][j]=(son[u][j] ? son[u][j] : 1);
            else trans[u][j]=(son[u][j] ? son[u][j] : trans[fail[u]][j]); 
        for(j=0;j<26;j++)       // lowercase only
            if(son[u][j]) que[++rear]=son[u][j];
    }
}

void match(char *str)
{
    int i,u,v;
    char x;
    for(u=1,i=0;str[i];i++)
    {
        x=str[i]-'a';

        for(v=u;v>1;v=fail[v])
            if(son[v][x])
            {
                mat[i]=son[v][x];
                break;
            }
        if(v==1) mat[i]=(son[1][x] ? son[1][x] : 1);

        mat[i]=trans[u][x];        // with array trans[][]

        u=mat[i];
    }
}
\end{lstlisting}

\subsection{Suffix array}
\begin{lstlisting}
char a[LENGTH];
int n,ofs,sa[LENGTH],hei[LENGTH],rk[LENGTH],tmp[LENGTH];

// radix sort
int c[ALPHABET];
void rsort()  // rk[]>0
{
    int b,i;
    for(b=1;b>=0;b--)
    {
        memset(c,0,sizeof(c));
        for(i=1;i<=n;i++)
            c[sa[i]+ofs*b<=n ? rk[sa[i]+ofs*b] : 0]++;
        for(i=1;i<=MAX_RANK;i++) c[i]+=c[i-1];
        for(i=n;i>=1;i--)
            tmp[c[sa[i]+ofs*b<=n ? rk[sa[i]+ofs*b] : 0]--]=sa[i];
        for(i=1;i<=n;i++) sa[i]=tmp[i];
    }
}

bool cmp(int i,int j)
{
    if(rk[i]!=rk[j]) return rk[i]<rk[j];
    int ri=(i+ofs<=n ? rk[i+ofs] : -1),rj=(j+ofs<=n ? rk[j+ofs] : -1);  // rank>=0
    return ri<rj;
}

void build_sa()
{
    int i;
    for(i=1;i<=n;i++)
    {
        sa[i]=i;
        rk[i]=a[i];
    }
    for(ofs=1;ofs<=n;ofs*=2)
    {
        sort(sa+1,sa+n+1,cmp);

        // rsort();
        // radix sort -- construct suffix array in O(nlogn)

        for(tmp[sa[1]]=1,i=2;i<=n;i++)
            tmp[sa[i]]=tmp[sa[i-1]]+(cmp(sa[i-1],sa[i]) ? 1 : 0);
        for(i=1;i<=n;i++) rk[i]=tmp[i];
    }
}

void build_hei()
{
    int h=0,i,j;
    for(i=1;i<=n;i++)
        rk[sa[i]]=i;
    for(i=1;i<=n;i++)
    {
        if(rk[i]==n)
        {
            hei[n]=0;
            continue;
        }
        j=sa[rk[i]+1];
        if(h) h--;
        for(;j+h<=n && i+h<=n;h++)
            if(a[j+h]!=a[i+h]) break;
        hei[rk[i]]=h;
    }
}

/* Count the number of distinct substrings:

#include <cstdio>
#include <cstring>
#include <algorithm>

typedef long long LL;
char a[100010];
int n,ofs,sa[100010],hei[100010],rk[100010],tmp[100010],l[100010],r[100010],stk[100010],last[100010];

// radix sort
int c[100010];
void rsort()  // rk[]>0
{
    int b,i;
    for(b=1;b>=0;b--)
    {
        memset(c,0,sizeof(c));
        for(i=1;i<=n;i++)
            c[sa[i]+ofs*b<=n ? rk[sa[i]+ofs*b] : 0]++;
        for(i=1;i<=std::max(n,200);i++) c[i]+=c[i-1];
        for(i=n;i>=1;i--)
            tmp[c[sa[i]+ofs*b<=n ? rk[sa[i]+ofs*b] : 0]--]=sa[i];
        for(i=1;i<=n;i++) sa[i]=tmp[i];
    }
}

bool cmp(int i,int j)
{
    if(rk[i]!=rk[j]) return rk[i]<rk[j];
    int ri=(i+ofs<=n ? rk[i+ofs] : -1),rj=(j+ofs<=n ? rk[j+ofs] : -1);  // rank>=0
    return ri<rj;
}

void build_sa()
{
    int i;
    for(i=1;i<=n;i++)
    {
        sa[i]=i;
        rk[i]=a[i];
    }
    for(ofs=1;ofs<=n;ofs*=2)
    {
        rsort();
        for(tmp[sa[1]]=1,i=2;i<=n;i++)
            tmp[sa[i]]=tmp[sa[i-1]]+(cmp(sa[i-1],sa[i]) ? 1 : 0);
        for(i=1;i<=n;i++) rk[i]=tmp[i];
    }
}

void build_hei()
{
    int h=0,i,j;
    for(i=1;i<=n;i++)
        rk[sa[i]]=i;
    for(i=1;i<=n;i++)
    {
        if(rk[i]==n)
        {
            hei[n]=0;
            continue;
        }
        j=sa[rk[i]+1];
        if(h) h--;
        for(;j+h<=n && i+h<=n;h++)
            if(a[j+h]!=a[i+h]) break;
        hei[rk[i]]=h;
    }
}

int main()
{
    scanf("%d%s",&n,a+1);
    build_sa();  build_hei();
    int i,j;
    for(j=0,i=1;i<=n;i++)
    {
        while(j && hei[i]<=hei[stk[j]]) j--;
        l[i]=j ? stk[j] : 0;
        stk[++j]=i;
    }
    for(j=0,i=n;i>=1;i--)
    {
        while(j && hei[i]<=hei[stk[j]]) j--;
        r[i]=j ? stk[j] : n+1;
        stk[++j]=i;
    }
    memset(last,-1,sizeof(last));
    LL ans=0;
    for(hei[0]=hei[n+1]=0,i=1;i<=n;i++)
    {
        ans+=n-sa[i]+1-std::max(hei[i-1],hei[i]);
        if(l[i]>last[hei[i]])
            ans+=hei[i]-std::max(hei[l[i]],hei[r[i]]);
            // cnt[i]=r[i]-l[i];
        last[hei[i]]=i;
    }
    printf("%lld\n",ans);
    return 0;
}
*/
\end{lstlisting}

\subsection{Suffix array (SA-IS)}
\begin{lstlisting}
#include <cstdio>
#include <cstring>
#include <algorithm>

char str[LENGTH];
int n,a[LENGTH*2],sa[LENGTH*2],typ[LENGTH*2],c[LENGTH+ALPHABET],p[LENGTH],sbuc[LENGTH+ALPHABET],lbuc[LENGTH+ALPHABET],name[LENGTH],hei[LENGTH],rk[LENGTH];

inline int islms(int *typ,int i)
{
    return !typ[i] && (i==1 || typ[i-1]);
}

int cmp(int *s,int *typ,int p,int q)
{
    do
    {
        if(s[p]!=s[q]) return 1;
        p++;  q++;
    }
    while(!islms(typ,p) && !islms(typ,q));
    return (!islms(typ,p) || !islms(typ,q) || s[p]!=s[q]);
}

void isort(int *s,int *sa,int *typ,int *c,int n,int m)
{
    int i;
    for(lbuc[0]=sbuc[0]=c[0],i=1;i<=m;i++)
    {
        lbuc[i]=c[i-1]+1;
        sbuc[i]=c[i];
    }
    for(i=1;i<=n;i++)
        if(sa[i]>1 && typ[sa[i]-1])
            sa[lbuc[s[sa[i]-1]]++]=sa[i]-1;
    for(i=n;i>=1;i--)
        if(sa[i]>1 && !typ[sa[i]-1])
            sa[sbuc[s[sa[i]-1]]--]=sa[i]-1;
}

void build_sa(int *s,int *sa,int *typ,int *c,int *p,int n,int m)    // the last character of s[] must be 0
{
    int i;
    for(i=0;i<=m;i++) c[i]=0;
    for(i=1;i<=n;i++) c[s[i]]++;
    for(i=1;i<=m;i++) c[i]+=c[i-1];
    typ[n]=0;
    for(i=n-1;i>=1;i--)
        if(s[i]<s[i+1]) typ[i]=0;
        else if(s[i]>s[i+1]) typ[i]=1;
        else typ[i]=typ[i+1];
    int cnt=0;
    for(i=1;i<=n;i++)
        if(!typ[i] && (i==1 || typ[i-1])) p[++cnt]=i;
    for(i=1;i<=n;i++) sa[i]=0;
    for(i=0;i<=m;i++) sbuc[i]=c[i];
    for(i=1;i<=cnt;i++)
        sa[sbuc[s[p[i]]]--]=p[i];
    isort(s,sa,typ,c,n,m);
    int last=0,t=-1,x;
    for(i=1;i<=n;i++)
    {
        x=sa[i];
        if(!typ[x] && (x==1 || typ[x-1]))
        {
            if(!last || cmp(s,typ,x,last))
                name[x]=++t;
            else name[x]=t;
            last=x;
        }
    }
    for(i=1;i<=cnt;i++)
        s[n+i]=name[p[i]];
    if(t<cnt-1) build_sa(s+n,sa+n,typ+n,c+m+1,p+cnt,cnt,t);
    else
        for(i=1;i<=cnt;i++)
            sa[n+s[n+i]+1]=i;
    for(i=0;i<=m;i++) sbuc[i]=c[i];
    for(i=1;i<=n;i++) sa[i]=0;
    for(i=cnt;i>=1;i--)
        sa[sbuc[s[p[sa[n+i]]]]--]=p[sa[n+i]];
    isort(s,sa,typ,c,n,m);
}

void build_hei()
{
    LL h=0,i,j;
    for(i=1;i<=n;i++)
        rk[sa[i]]=i;
    for(i=1;i<=n;i++)
    {
        if(rk[i]==n)
        {
            hei[n]=0;
            continue;
        }
        j=sa[rk[i]+1];
        if(h) h--;
        for(;j+h<=n && i+h<=n;h++)
            if(a[j+h]!=a[i+h]) break;
        hei[rk[i]]=h;
    }
}

int main()
{
    scanf("%s",str);
    n=strlen(str);
    int i;
    for(i=1;i<=n;i++)
        a[i]=str[i-1];
    a[++n]=0;               // the last character of a[] must be 0
    build_sa(a,sa,typ,c,p,n,200);
    for(i=2;i<=n;i++)
        printf("%d%s",sa[i],i<n ? " " : "\n");
    return 0;
}

/* Count the number of distinct substrings:

#include <cstdio>
#include <cstring>
#include <algorithm>

typedef long long LL;
const int LENGTH=100050,ALPHABET=100050;
char str[LENGTH];
int n,a[LENGTH*2],sa[LENGTH*2],typ[LENGTH*2],c[LENGTH+ALPHABET],p[LENGTH],sbuc[LENGTH+ALPHABET],lbuc[LENGTH+ALPHABET],name[LENGTH],hei[LENGTH],rk[LENGTH];
int l[100010],r[100010],stk[100010],last[100010];

inline int islms(int *typ,int i)
{
    return !typ[i] && (i==1 || typ[i-1]);
}

int cmp(int *s,int *typ,int p,int q)
{
    do
    {
        if(s[p]!=s[q]) return 1;
        p++;  q++;
    }
    while(!islms(typ,p) && !islms(typ,q));
    return (!islms(typ,p) || !islms(typ,q) || s[p]!=s[q]);
}

void isort(int *s,int *sa,int *typ,int *c,int n,int m)
{
    int i;
    for(lbuc[0]=sbuc[0]=c[0],i=1;i<=m;i++)
    {
        lbuc[i]=c[i-1]+1;
        sbuc[i]=c[i];
    }
    for(i=1;i<=n;i++)
        if(sa[i]>1 && typ[sa[i]-1])
            sa[lbuc[s[sa[i]-1]]++]=sa[i]-1;
    for(i=n;i>=1;i--)
        if(sa[i]>1 && !typ[sa[i]-1])
            sa[sbuc[s[sa[i]-1]]--]=sa[i]-1;
}

void build_sa(int *s,int *sa,int *typ,int *c,int *p,int n,int m)    // the last character of s[] must be 0
{
    int i;
    for(i=0;i<=m;i++) c[i]=0;
    for(i=1;i<=n;i++) c[s[i]]++;
    for(i=1;i<=m;i++) c[i]+=c[i-1];
    typ[n]=0;
    for(i=n-1;i>=1;i--)
        if(s[i]<s[i+1]) typ[i]=0;
        else if(s[i]>s[i+1]) typ[i]=1;
        else typ[i]=typ[i+1];
    int cnt=0;
    for(i=1;i<=n;i++)
        if(!typ[i] && (i==1 || typ[i-1])) p[++cnt]=i;
    for(i=1;i<=n;i++) sa[i]=0;
    for(i=0;i<=m;i++) sbuc[i]=c[i];
    for(i=1;i<=cnt;i++)
        sa[sbuc[s[p[i]]]--]=p[i];
    isort(s,sa,typ,c,n,m);
    int last=0,t=-1,x;
    for(i=1;i<=n;i++)
    {
        x=sa[i];
        if(!typ[x] && (x==1 || typ[x-1]))
        {
            if(!last || cmp(s,typ,x,last))
                name[x]=++t;
            else name[x]=t;
            last=x;
        }
    }
    for(i=1;i<=cnt;i++)
        s[n+i]=name[p[i]];
    if(t<cnt-1) build_sa(s+n,sa+n,typ+n,c+m+1,p+cnt,cnt,t);
    else
        for(i=1;i<=cnt;i++)
            sa[n+s[n+i]+1]=i;
    for(i=0;i<=m;i++) sbuc[i]=c[i];
    for(i=1;i<=n;i++) sa[i]=0;
    for(i=cnt;i>=1;i--)
        sa[sbuc[s[p[sa[n+i]]]]--]=p[sa[n+i]];
    isort(s,sa,typ,c,n,m);
}

void build_hei()
{
    LL h=0,i,j;
    for(i=1;i<=n;i++)
        rk[sa[i]]=i;
    for(i=1;i<=n;i++)
    {
        if(rk[i]==n)
        {
            hei[n]=0;
            continue;
        }
        j=sa[rk[i]+1];
        if(h) h--;
        for(;j+h<=n && i+h<=n;h++)
            if(a[j+h]!=a[i+h]) break;
        hei[rk[i]]=h;
    }
}

int main()
{
    scanf("%d%s",&n,str);
    int i,j;
    for(i=1;i<=n;i++) a[i]=str[i-1];
    a[++n]=0;
    build_sa(a,sa,typ,c,p,n,200);
    build_hei();
    for(j=0,i=1;i<=n;i++)
    {
        while(j && hei[i]<=hei[stk[j]]) j--;
        l[i]=j ? stk[j] : 0;
        stk[++j]=i;
    }
    for(j=0,i=n;i>=1;i--)
    {
        while(j && hei[i]<=hei[stk[j]]) j--;
        r[i]=j ? stk[j] : n+1;
        stk[++j]=i;
    }
    memset(last,-1,sizeof(last));
    LL ans=0;
    for(hei[0]=hei[n+1]=0,i=1;i<=n;i++)
    {
        ans+=n-sa[i]-std::max(hei[i-1],hei[i]);
        if(l[i]>last[hei[i]])
            ans+=hei[i]-std::max(hei[l[i]],hei[r[i]]);
            // cnt[i]=r[i]-l[i];
        last[hei[i]]=i;
    }
    printf("%lld\n",ans);
    return 0;
}
*/
\end{lstlisting}

\subsection{Manacher}
\begin{lstlisting}
void manacher()
{
    LL i,j,k;
    for(i=2,j=1,r[1]=0;i<=len;i++)
    {
        if(i<=j+r[j]) r[i]=min(r[j*2-i],j+r[j]-i);
        else r[i]=0;
        if(i+r[i]<j+r[j]) continue;
        for(k=r[i]+1;i+k<=len && i-k>=1;k++)
            if(str[i+k]==str[i-k]) r[i]++;
            else break;
        j=i;
    }
}
\end{lstlisting}

\subsection{Palindromic automaton}
\begin{lstlisting}
int sz,son[N_NODE][ALPHABET],len[N_NODE],fail[N_NODE],cnt[N_NODE],pos[N_NODE];

void addnode(int l)
{
    sz++;
    memset(son[sz],0,sizeof(son[sz]));
    len[sz]=l;  cnt[sz]=0;
}

void build(const string &str)
{
    sz=0;
    addnode(-1);  addnode(0);
    fail[sz-1]=fail[sz]=sz-1;
    int i,u,x,l=str.size();
    for(u=1,i=1;i<=l;i++)
    {
        while(i-len[u]-1<1 || str[i-len[u]-1-1]!=str[i-1]) u=fail[u];
        x=str[i-1]-'a';
        if(!son[u][x])
        {
            addnode(len[u]+2);
            int v=fail[u];
            while(i-len[v]-1<1 || str[i-len[v]-1-1]!=str[i-1]) v=fail[v];
            fail[sz]=(son[v][x] ? son[v][x] : 2);
            son[u][x]=sz;
        }
        cnt[son[u][x]]++;
        u=son[u][x];
    }
    for(i=sz;i>3;i--)
        cnt[fail[i]]+=cnt[i];
}

void match(const string &str)
{
    int i,u,x,l=str.size();
    for(u=1,i=1;i<=l;i++)
    {
        x=str[i-1]-'a';
        while(i-len[u]-1<1 || str[i-len[u]-1-1]!=str[i-1] || u>1 && !son[u][x]) u=fail[u];
        u=son[u][x] ? son[u][x] : 1;
        pos[i]=u;
    }
}
\end{lstlisting}


\section{Geometry}
\subsection{2-D vector}
\begin{lstlisting}
inline int dcmp(double x,double y)
{
    if(x>y+EPS) return 1;
    else if(x>y-EPS) return 0;
    else return -1;
}
    
struct V
{
    double x,y;
    // the operators are sorted by their priorities
    double operator*(const V &rhs) const    // dot product
    {
        return x*rhs.x+y*rhs.y;
    }
    V operator/(double k) const    // scalar division
    {
        return V{x/k,y/k};
    }
    V operator+(const V &rhs) const
    {
        return V{x+rhs.x,y+rhs.y};
    }
    V operator-(const V &rhs) const
    {
        return V{x-rhs.x,y-rhs.y};
    }
    bool operator<(const V &rhs) const      // sort vectors by polar angle ([0, 2pi))
    {
        return arg()<rhs.arg();
    }
    V operator&(double k) const    // scalar multiplicalion
    {
        return V{x*k,y*k};
    }
    double operator^(const V &rhs) const    // cross product
    {
        return x*rhs.y-y*rhs.x;
    }
    bool operator||(const V &rhs) const     // in the same direction
    {
        return !dcmp(*this^rhs,0) && dcmp(*this*rhs,0)>0;
    }
    double len() const
    {
        return sqrt(x*x+y*y);
    }
    double arg() const      // polar angle ([0, 2pi))
    {
        double t=atan2(y,x);
        return dcmp(t,0)<0 ? t+PI*2 : t;
    }
    V rot(double alp) const     // rotate counterclockwise
    {
        return V{x*cos(alp)-y*sin(alp),x*sin(alp)+y*cos(alp)};
    }
};
\end{lstlisting}

\subsection{2-D line}
\begin{lstlisting}
inline int dcmp(double x,double y)
{
    if(x>y+EPS) return 1;
    else if(x>y-EPS) return 0;
    else return -1;
}
        
struct LN
{
    V p,q;
    operator V() const
    {
        return q-p;
    }
    bool operator<(const LN &rhs) const      // sort lines by polar angle ([0, 2pi))
    {
        if((q-p)||rhs)
            return dcmp((rhs.p-p)^(rhs.q-p),0)>0;
        else return (q-p)<rhs;
    }
    V inters(const LN &rhs) const
    {
        if(!dcmp((q-p)^rhs,0)) return V{INF,INF};
        V t1=*this,t2=rhs,t3=rhs.p-p;
        return p+(t1&((t3^t2)/(t1^t2)));
    }
    LN shift(double l) const
    {
        V d=q-p;
        d=d/d.len();
        V n=d.rot(PI/2);
        return LN{p+(n&l),q+(n&l)};
    }
};
\end{lstlisting}

\subsection{Area of polygon}
\begin{lstlisting}
// Include struct V (2D vector) and struct LN (2D line).

double area(V *arr,int n)      // array of vertex
{
    int i;
    double res=0;
    for(i=2;i<n;i++)
        res+=(arr[i]-arr[1])^(arr[i+1]-arr[1])/2;
    return fabs(res);
}
    
V tmp[N_VERTEX];
double area(LN *ln,int n)      // array of lines
{
    int i;
    for(i=1;i<=n;i++)
        tmp[i]=ln[i].inters(ln[i%n+1]);
    double res=0;
    for(i=2;i<n;i++)
        res+=(tmp[i]-tmp[1])^(tmp[i+1]-tmp[1])/2;
    return fabs(res);
}
\end{lstlisting}

\subsection{Point-polygon distance}
\begin{lstlisting}
double distance(V o,LN *arr,int n)
{
    double res=INF;
    int i;
    for(i=1;i<=n;i++)
        if(dcmp(((arr[i].p-o)*arr[i])*((arr[i].q-o)*arr[i]),0)>0)
            res=std::min(res,std::min((arr[i].p-o).len(),(arr[i].q-o).len()));
        else res=std::min(res,fabs((arr[i].p-o)^(arr[i].q-o))/V(arr[i]).len());
    return res;
}
\end{lstlisting}

\subsection{Segment+circle intersection}
\begin{lstlisting}
// Include struct V (2D vector).

int inters(V p,V q,V o,double r,V *res)
{
    double dx=q.x-p.x,dy=q.y-p.y;
    double a=sqr(dx)+sqr(dy),b=2*dx*(p.x-o.x)+2*dy*(p.y-o.y),c=sqr(p.x-o.x)+sqr(p.y-o.y)-sqr(r);
    double delta=sqr(b)-4*a*c;
    int cnt=0;
    if(delta>EPS)
    {
        double t1=(-b+sqrt(delta))/a/2,t2=(-b-sqrt(delta))/a/2;
        if(t1>-EPS && t1<1+EPS)     ////
            res[++cnt]=V{p.x+t1*dx,p.y+t1*dy};
        if(t2>-EPS && t2<1+EPS)     ////
            res[++cnt]=V{p.x+t2*dx,p.y+t2*dy};
        return cnt;
    }
    else if(delta>-EPS)     // delta=0
    {
        double t=-b/a/2;
        if(t>-EPS && t<1+EPS)       // To find line+circle intersections, delete the 3 'if's.
            res[++cnt]=V{p.x+t*dx,p.y+t*dy};
        return cnt;
    }
    else return 0;
}
\end{lstlisting}

\subsection{Half-plane intersection}
\begin{lstlisting}
// Include struct V (2D vector) and struct LN (2D line).

inline int dcmp(double x,double y)
{
    if(x>y+EPS) return 1;
    else if(x>y-EPS) return 0;
    else return -1;
}

inline bool check(const LN &last,const LN &last2,const LN &l)
{
    V crs=last.inters(last2);
    return dcmp((l.p-crs)^(l.q-crs),0)>0;
}

int hp_inters(LN *arr,int n,LN *res)    // the intersection must be a polygon!
{
    std::sort(arr+1,arr+n+1);
    int i,front=1,rear=0;
    for(i=1;i<=n;i++)
    {
        if(front<=rear && (V(h[rear])||arr[i])) continue;
        while(front<rear && !check(h[rear],h[rear-1],arr[i])) rear--;
        while(front<rear && !check(h[front],h[front+1],arr[i])) front++;
        h[++rear]=arr[i];
    }
    while(front<rear && !check(h[rear],h[rear-1],h[front])) rear--;
    int m=0;
    for(i=front;i<=rear;i++) res[++m]=h[i];
    return m;
}
\end{lstlisting}

\subsection{Convex hull}
\begin{lstlisting}
#include <cstdio>
#include <cstring>
#include <algorithm>
#include <cmath>

using namespace std;
const double PI=acos(-1);
struct VERTEX
{
    int x,y;
    VERTEX(){}
    VERTEX(int xx,int yy)
    {
        x=xx,y=yy;
    }
    bool operator<(const VERTEX &rhs) const
    {
        if(x!=rhs.x) return x<rhs.x;
        else return y<rhs.y;
    }
    VERTEX operator-(const VERTEX &rhs) const
    {
        return VERTEX(x-rhs.x,y-rhs.y);
    }
}
v[1010];
int n,h[2010];
inline double dist(const VERTEX &v1,const VERTEX &v2)
{
    return sqrt((v1.x-v2.x)*(v1.x-v2.x)+(v1.y-v2.y)*(v1.y-v2.y));
}
inline int cross(const VERTEX &v1,const VERTEX &v2)
{
    return v1.x*v2.y-v1.y*v2.x;
}

int main()
{
    scanf("%d",&n);
    int i;
    for(i=1;i<=n;i++)
        scanf("%d%d",&v[i].x,&v[i].y);
    sort(v+1,v+n+1);
    int sz=0,p=0;
    for(i=1;i<=n;i++)
    {
        while(p>=sz+2 && cross(v[h[p]]-v[h[p-1]],v[i]-v[h[p]])<0) p--;
        h[++p]=i;
    }
    sz=--p;
    for(i=n;i>=1;i--)
    {
        while(p>=sz+2 && cross(v[h[p]]-v[h[p-1]],v[i]-v[h[p]])<0) p--;
        h[++p]=i;
    }
    sz=--p;
    double ans=0;
    for(i=1;i<sz;i++)
        ans+=dist(v[h[i]],v[h[i%sz+1]]);  // ans is the circumference of the hull
    printf("%.0f\n",ans);
    return 0;
}
\end{lstlisting}

\subsection{Rotating calipers}
\begin{lstlisting}
int n;
V v[N_VERTEX*2];

inline double area(int i,int j,int k)
{
    return fabs(v[i].x*v[j].y+v[j].x*v[k].y+v[k].x*v[i].y-v[j].x*v[i].y-v[k].x*v[j].y-v[i].x*v[k].y);
}

double maxarea()
{
    int i,j,k;
    double ans=0;
    for(i=1;i<=n;i++) v[i+n]=v[i];
    for(i=1;i<=n;i++)
        for(k=i+2,j=i+1;j<i+n-1;j++)
        {
            k=max(k,j+1);
            while(k<i+n-1 && dcmp(area(i,j,k),area(i,j,k+1))<0) k++;
            ans=max(ans,area(i,j,k));
        }
    return ans;
}
\end{lstlisting}


\section{IO}
\subsection{Fast read}
\begin{lstlisting}
const int BUF_SZ=100000;
char buf[BUF_SZ+10];
inline char nc(void)
{
    static char *pr=buf,*pend=buf;
    if(pr==pend)
    {
        pr=buf;
        pend=pr+fread(buf,1,BUF_SZ,stdin);
        if(pr==pend) return EOF;
        else return *pr++;
    }
    return *pr++;
}
inline int readint(int *x)
{
    static char ch;
    ch=nc();
    while(ch!=EOF && (ch<'0' || ch>'9')) ch=nc();
    if(ch==EOF) return 0;
    for(*x=0;ch>='0' && ch<='9';ch=nc())
        *x=*x*10+ch-'0';
    return 1;
}

/*when reading from files, use "rb" mode!!!*/
\end{lstlisting}

\subsection{Fast input (fin)}
\begin{lstlisting}
/*when reading from files, use "rb" mode!!!*/

class Fast_in
{
private:
    static const int BUF_SZ=100000;
    char buf[BUF_SZ+10],*pr,*pend,ch;
    bool state;
    int len;
    char nc()
    {
        if(pr==pend)
        {
            pr=buf;
            pend=pr+fread(buf,1,BUF_SZ,stdin);
            if(pr==pend) return EOF;
            else return *pr++;
        }
        return *pr++;
    }
    bool isblank(char ch)
    {
        if(ch==' ' || ch=='\t' || ch=='\n' || ch=='\r') return true;
        else return false;
    }
    bool isdigit(char ch)
    {
        if(ch>='0' && ch<='9') return true;
        else return false;
    }
public:
    Fast_in()
    {
        pr=pend=buf;
        state=true;
    }
    operator bool()
    {
        return state;
    }
    Fast_in& get(char &ch)
    {
        state=true;
        ch=nc();
        if(ch==EOF) state=false;
        return *this;
    }

    Fast_in& getline(char *s,int n,char delim='\n')
    {
        state=true;
        for(len=0,ch=nc();len<n-1 && ch!=EOF && ch!=delim;len++,ch=nc()) *s++=ch;
        *s++='\0';
        if(ch==EOF) state=false;
        else if(len==n-1) *--pr=ch;
        return *this;
    }

    friend inline Fast_in& operator>>(Fast_in &fin,char *s)
    {
        static char ch;
        fin.state=true;
        do ch=fin.nc();
        while(fin.isblank(ch));
        if(ch==EOF)
        {
            fin.state=false;
            return fin;
        }
        for(;ch!=EOF && !fin.isblank(ch);ch=fin.nc()) *s++=ch;
        *s++='\0';
        if(ch!=EOF) *--fin.pr=ch;
        return fin;
    }

    template<class T>
    friend inline Fast_in& operator>>(Fast_in &fin,T &x)
    {
        static char ch;
        static int sig;
        fin.state=true;
        do ch=fin.nc();
        while(fin.isblank(ch));
        if(ch!='-' && !fin.isdigit(ch))
        {
            fin.state=false;
            if(ch!=EOF) *--fin.pr=ch;
            return fin;
        }
        if(ch=='-')
        {
            sig=-1;
            ch=fin.nc();
        }
        else sig=1;
        for(x=0;fin.isdigit(ch);ch=fin.nc())
            x=x*10+ch-'0';
        x*=sig;
        if(ch!=EOF) *--fin.pr=ch;
        return fin;
    }
}fin;
\end{lstlisting}
\end{document}